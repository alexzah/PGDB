\documentclass{refart}

\usepackage{hyperref}
\usepackage{fancyvrb}
% Set the indent for the verbatim section.
\fvset{xleftmargin=0.5cm}
\usepackage{enumitem}
\usepackage{makeidx}

\newcommand{\lmargintt}[1]{\marginlabel{\texttt{#1}} \index{#1}}
\newcommand{\ttemph}[1]{\texttt{\emph{#1}}}
\newcommand{\deprecated}[1]{\textbf{Deprected:} #1\newline}

\makeindex

\title{PGDB Manual}
\author{Nikoli Dryden}
\date{\today}

\begin{document}

\maketitle

\newpage

\setcounter{page}{1}
\pagenumbering{roman}
\tableofcontents

\newpage

\setcounter{page}{1}
\pagenumbering{arabic}

\section{Introduction}

PGDB is the ``Parallel GDB''.

PGDB is a parallelized version of GDB, the GNU Debugger, designed for debugging large-scale, MPI (message passing interface) programs in HPC (high-performance computing) environments.

PGDB aims to be familiar to users of vanilla GDB, maintaining the majority of its interface, command semantics, and functionality. It can be succinctly described as a convenient interface to ``GDB on a bunch of machines simultaneously''. Because of being built upon GDB, PGDB supports debugging the same set of targets as GDB does, and typically, if something works in GDB, it will work in PGDB as well.

PGDB was originally developed to facilitate debugging of the Kull project at Lawrence Livermore National Laboratory, and this influences its functionality. In particular, PGDB is well-suited to debugging large scientific simulation codes: it has little trouble with very large C++ programs that make extensive use of template meta-programming, polymorphism, inheritance, and STL containers, with many hundreds of libraries written in C, C++, and Fortran.

Additionally, PGDB aims to be scalable and lightweight compared to debuggers like TotalView.

Please note that, at present, PGDB is still in development, and there may be bugs, missing or incomplete features, or other rough edges. If you come across one, please report it.

This manual is laid out as follows:

\begin{itemize}
\item Chapter 2 covers the installation and initial configuration process for setting up PGDB in detail.
\item Chapter 3 covers the start-up and basic usage of PGDG.
\item Chapter 4 includes many examples of commonly-used functionality in real situations to illustrate how to use PGDB.
\item Chapter 5 is a complete reference to the commands available within PGDB.
\item Chapter 6 is a complete reference to the configuration options supported within PGDB's configuration files.
\end{itemize}

\newpage

\section{Installation and Configuration}

This chapter details the installation process and subsequent configuration in order to get PGDB running.

\subsection{Supported Platforms}

PGDB is designed to be portable, and in theory should work on any platform that supports its requirements.

At present, PGDB has been tested on various Linux clusters, including those at Lawrence Livermore National Laboratory and the University of Illinois at Urbana-Champaign. It is likely that PGDB will work on any reasonable UNIX-based distribution with little trouble.

Support for other platforms is not available at present, but planned for in the future.

PGDB supports the following MPI distributions: OpenMPI (at least 1.5), MPICH2, MVAPICH, Slurm, ALPS (untested), and Blue Gene systems (untested).

\subsection{Obtaining PGDB}

PGDB may be obtained from its project page at \url{https://github.com/ndryden/PGDB}.

\subsection{Requirements}

The following are necessary for PGDB:

\begin{itemize}
\item Python 2.6 or greater

  \url{http://www.python.org}
\item GDB 7.0 or greater; 7.4 or greater recommended

  \url{http://sources.redhat.com/gdb/}
\item LaunchMON 0.7.2 or 1.0; 1.0 recommended

  \url{http://sourceforge.net/projects/launchmon/}
\item MRNet 4.0.0

  \url{http://www.paradyn.org/mrnet/}
\item PyBindGen 0.16.0 or greater (only needed for regenerating MRNet bindings)

  \url{http://code.google.com/p/pybindgen/}
\end{itemize}

This excludes the dependencies of the above. Additionally, Python bindings for LaunchMON and MRNet must be built and installed. These are included in the PGDB distribution and this process is documented below.

\subsection{Installation}

Installation of PGDB requires installing its dependencies, installing PGDB, and configuring PGDB. This process is presently completely manual and rather tedious; it is being automated. Documentation on this is additionally maintained on the project website, available at \url{https://github.com/ndryden/PGDB/wiki/Installation}. Check this page if you have problems following the procedure herein, as it may be more up-to-date.

\subsubsection{Installing Dependencies}

Python and GDB should either already be installed on your system, or should be easy to install. Refer to their documentation for further details. This section deals with only the installation of LaunchMON, MRNet, and PGDB's bindings for these, as external documentation on these is sparse and the typical installation process frequently fails.

You will need all the standard tools for building packages from source, including a C and C++ compiler, automake, autoconf, and libtool. You will need Git and Subversion for checking out packages. The dependencies additionally have the following dependencies: libelf, libboost, libgpg-error, and ncurses (you will need the development versions of these). A remote shell will be required; depending upon your situation, either rsh or ssh can be used. An MPI distribution is additionally required; see above for details.

The latest version of LaunchMON should be installed; at the time of this writing, while there is a stable version 0.7.2 available, it is recommended to use the release branch of the 1.0 version (\texttt{launchmon-1.0-release}), available from LaunchMON's Subversion repository. LaunchMON requires libelf, which can be installed either via your platform package manager or obtained from \url{http://www.mr511.de/software/english.html}. Due to current problems in building LaunchMON, some older versions of tools are required: gcc-4.4, g++-4.4, and binutils 2.20.1a. Your system package manager should be able to install gcc-4.4 and g++-4.4. Once this is done, ensure that it is the default compiler (e.g., by checking what \texttt{/usr/bin/gcc} and \texttt{/usr/bin/g++} points to and updating the symlink if needed).

Binutils 2.20.1a can be obtained at \url{http://ftp.gnu.org/gnu/binutils/binutils-2.20.1a.tar.bz2}, and can be built from source without problems. It is recommended that this not be installed in the main system path, to avoid polluting your environment. Once this is done, you can proceed with installing LaunchMON as follows:

\begin{itemize}
\item Ensure that your \texttt{PATH} and \texttt{LD\_LIBRARY\_PATH} include your MPI distribution if it is not in a default path, as well as the old version of binutils installed above.
\item Change to the \texttt{launchmon} directory (\texttt{cd launchmon}).
\item Edit \texttt{launchmon/src/linux/Makefile.am}, on the line beginning with \texttt{launchmon\_LDADD}, append \texttt{-lgpg-error}.
\item Edit \texttt{tools/Makefile.am}, change the line with \texttt{OPT\_GPG = libgpg-error} to \texttt{OPT\_GPG = }.
\item Begin by running \texttt{./bootstrap} to ensure the configure scripts are suitable for the platform.
\item Run \texttt{./configure --prefix=\emph{path}} and replace \emph{path} with the path to install LaunchMON to.
\item Run \texttt{make}. If your MPI installation is not in a standard location, instead run \texttt{CFLAGS="-I\emph{path}"} where \emph{path} is the path to the MPI \texttt{include} directory.
\item Run \texttt{make install}.
\end{itemize}

To install MRNet, download and extract it, then change to the extracted directory and:

\begin{itemize}
\item Edit \texttt{conf/install\_link.sh} and change \texttt{\#!/bin/sh} to \texttt{\#!/bin/bash} (this is due to some systems changing the default shell).
\item Run \texttt{./configure --prefix=\emph{path}} and replace \emph{path} with the path to install MRNet to.
\item Run \texttt{make}.
\item Run \texttt{make install}.
\item Change to the install directory. Due to problems with the installation script, you will need to move around a few files.
\item Run \texttt{mv lib/mrnet-4.0.0/include/mrnet\_config.h include/mrnet/} then \texttt{rm -rf lib/mrnet-4.0.0}.
\item Run \texttt{mv lib/xplat-4.0.0/include/xplat\_config.h include/xplat/} then \texttt{rm -rf lib/xplat-4.0.0}.
\end{itemize}

At this point, you will need to check out PGDB from the Git repository, then configure and install the MRNet and LaunchMON bindings. This process is currently in flux and being simplified, so please refer to the online installation guide for the latest information, \url{https://github.com/ndryden/PGDB/wiki/Installation}.

If you need to manually regenerate the MRNet bindings, the process is as follows, from the \texttt{mrnet} directory. You will need PyBindGen installed. Run:

\begin{Verbatim}
python mrnetbind.py > mrnet_module.cpp
\end{Verbatim}

to generate the bindings and place them in the \texttt{mrnet\_module.cpp} file.

\subsubsection{Installing PGDB}

PGDB is simply a collection of Python scripts and may be placed anywhere that is desired provided the directory structure maintains intact. For example, if the PGDB directory is \texttt{pgdb} and you wish to place it in \texttt{/usr/local}, use

\begin{Verbatim}
mv pgdb /usr/local/
\end{Verbatim}

You should ensure that whichever directory you choose is in your \texttt{\$PATH} environment variable for easy use.

You may wish to pre-compile the Python files to corresponding `\texttt{.pyc}' files, both for performance and to prevent file permissions errors. This can be done with the following command from the main PGDB directory:
\begin{Verbatim}
python -m compileall ./
\end{Verbatim}

\subsubsection{Basic Configuration}

\textbf{Note:} This section is currently somewhat out-of-date. The information needed to install PGDB is in the online guide referenced in the prior section. This is preserved for reference and will be updated in the future.

The configuration process for PGDB involves editing two configuration files, both of which are located in the \texttt{conf} directory. The configuration files are regular Python scripts and may be treated as such if desired. Both files contain many comments inside them to provide assistance.

\lmargintt{use\_lmon\_10}
First we set up the LaunchMON configuration in \texttt{lmonconf.py}. First PGDB must be told whether to use LaunchMON 1.0 or 0.7.2. This must match the version of LaunchMON that was installed before. To use LaunchMON 1.0, we have

\begin{Verbatim}
use_lmon_10 = True
\end{Verbatim}

and to use 0.7.2

\begin{Verbatim}
use_lmon_10 = False
\end{Verbatim}

For the remainder of the LaunchMON configuration, ensure that you are editing the section corresponding to the LaunchMON version you specified.

\lmargintt{lmon\_fe\_lib}
\lmargintt{lmon\_be\_lib}
We now need to provide the locations for the LaunchMON front- and back-end libraries. The front-end library will be named \texttt{lmonfeapi.so} and the back-end library \texttt{lmonbeapi.so}. We have

\begin{Verbatim}
lmon_fe_lib = "/path/to/lmonfeapi.so"
lmon_be_lib = "/path/to/lmonbeapi.so"
\end{Verbatim}

The \texttt{lmon\_version} variable should not be edited.

\lmargintt{lmon\_environ}
Finally, we need to configure the environment for LaunchMON. The variable \texttt{lmon\_environ} is a Python dictionary; the keys will become environment variables with the corresponding value. Three environment varaibles need to be set.

\lmargintt{LMON\_REMOTE\_LOGIN}
We need to specify the command used for logging in to remote nodes. Two options are available: \texttt{rsh} and \texttt{ssh}. It is recommended to use \texttt{rsh}. While it is not required, you should provide the full path to the binary.

\begin{Verbatim}
"LMON_REMOTE_LOGIN": "/usr/bin/rsh"
\end{Verbatim}

\lmargintt{LMON\_PREFIX}
This is the path to the top-level LaunchMON install directory and should be set based on where LaunchMON was installed.

\begin{Verbatim}
"LMON_PREFIX": "/path/to/launchmon"
\end{Verbatim}

\lmargintt{LMON\_LAUNCHMON\_ENGINE\_PATH}
This is the path to the \texttt{launchmon} binary and should be set based on where LaunchMON was installed.

\begin{Verbatim}
"LMON_LAUNCHMON_ENGINE_PATH": "/path/to/bin/launchmon"
\end{Verbatim}

When you are done with this, you should have a section that looks something like this

\begin{Verbatim}
lmon_environ = {"LMON_REMOTE_LOGIN": "/usr/bin/rsh",
                "LMON_PREFIX": "/path/to/launchmon",
                "LMON_LAUNCHMON_ENGINE_PATH": "/path/to/bin/launchmon"}
\end{Verbatim}

Next, we set up the main configuration in \texttt{gdbconf.py}. There are many configuration settings, but we only concern ourselves with a few at this point. Again, there are many comments in the file to guide you.

\lmargintt{\$PYTHONPATH}
If needed, this script can modify the path Python uses to load modules. If one or more of the Python modules was installed in a location that is not in the Python path by default, use the following to add the path:

\begin{Verbatim}
import sys
sys.path.append("/path/to/module/directory/")
\end{Verbatim}

Multiple paths can be added with multiple \texttt{append} commands.

\lmargintt{backend\_bin}
The binary for the back-end daemons needs to be specified. It defaults to the value \texttt{"python"}, which should be suitable, but if the version of Python you wish to use is installed in a different location, specify the path to the binary here.

\begin{Verbatim}
backend_bin = "/path/to/python"
\end{Verbatim}

\lmargintt{backend\_args}
The arguments to the back-end daemons should be specified as a list. The first argument must always be the path to the \texttt{gdbbe.py} script that was installed, and should be configured appropriately. Other arguments may be added following this.

\begin{Verbatim}
backend_args = ["/path/to/gdbbe.py"]
\end{Verbatim}

\lmargintt{environ}
The environment for GDB and MRNet needs to be configured. The process is similar to the one for LaunchMON. Three environment variables need to be specified.

\lmargintt{XPLAT\_RSH}
This is the remote shell command to use; again, either \texttt{rsh} or \texttt{ssh} may be used, but \texttt{rsh} is recommended.

\begin{Verbatim}
environ["XPLAT_RSH"] = "rsh"
\end{Verbatim}

\lmargintt{MRNET\_COMM\_PATH}
The path to the MRNet \texttt{mrnet\_commnode} binary needs to be specified based on where MRNet was installed.

\begin{Verbatim}
environ["MRNET_COMM_PATH"] = "/path/to/bin/mrnet_commnode"
\end{Verbatim}

\lmargintt{LD\_LIBRARY\_PATH}
This is an optional configuration setting that specifies the directory containing the MRNet libraries. It should be specified if the libraries are not found in the default system library search paths.

\begin{Verbatim}
environ["LD_LIBRARY_PATH"] = "/path/to/mrnet/lib"
\end{Verbatim}

\lmargintt{gdb\_init\_path}
Lastly, the path to the GDB initialization file needs to be specified. This should be in the directory that PGDB was installed in, named \texttt{gdbinit}. Set the path accordingly.

\begin{Verbatim}
gdb_init_path = "/path/to/pgdb/gdbinit"
\end{Verbatim}

For more details on all configuration options, see chapter 5.

\subsection{Updating Pretty-printers}

PGDB has GDB load Python pretty-printers to facilitate the printing of STL container data types. The \texttt{gdbinit} file takes care of loading these. However, you \emph{must} use a version of the pretty-printers that supports the version of the STL the programs you are debugging were built with.

PGDB includes pretty-printers from GCC 4.6, which are known to also work with code compiled with GCC 4.4. These can be obtained from the GCC Subversion repository at \url{svn://gcc.gnu.org/svn/gcc/branches/gcc-4_6-branch/libstdc++-v3/python}. If you are using a newer version than 4.6, or a different compiler, you will need to obtain different pretty-printers.

\newpage

\section{Usage}

There are two ways to use PGDB: attach mode and launch mode. PGDB works the same after the startup completes, the only difference is how it obtains control of the target application for debugging. Attach mode debugs running processes whereas launch mode launches a new process under debug control.

\subsection{Attach Mode}

Attach mode should be used when you wish to debug a process that is already running. PGDB will deploy debuggers to every node on which the process is running, stop the processes, attach GDB to each process, and be available for debugging. The process is simple and involves only the PID of the \texttt{srun} process that launched the job.

\begin{Verbatim}
pgdb -p PID
\end{Verbatim}

where \texttt{PID} is the process ID of the corresponding \texttt{srun} process.

\subsection{Launch Mode}

Launch mode should be used when you wish to have a process under debug control from the very beginning. PGDB will launch job as specified, deploy debuggers to every node on which the job will run, stop the processes at their entry point, attach GDB to each process, and be available for debugging. The process is similar to launching a job with \texttt{srun}.

\begin{Verbatim}
pgdb [--launcher launcher] -a args
\end{Verbatim}

where \texttt{args} are the arguments you would normally pass to \texttt{srun} to launch the job and optionally \texttt{launcher} is the launcher to use, which defaults to \texttt{srun}. Note that everything provided after a \texttt{-a} is passed verbatim to the launcher.

\subsection{Local Mode}

Both of the above modes are designed for use on a cluster where PGDB is launched from a node on which the job is \emph{not} running. If this is not the case (e.g., you are debugging a job running on a single node), the \texttt{--local} argument can be given to either of the above modes to indicate this. PGDB will fail if it is not in local mode and its front-end and back-ends are deployed to the same machine.

\subsection{Interaction}

After starting the debugger, you will first get a report that ``GDB has been deployed to $x$ hosts and $y$ processors.'' This indicates that the debugger has successfully deployed and is now attaching to the target processes. Once this process is complete, you should, for each processor, see a ``Stopped.'' message, indicating that the application is stopped and ready for debugging. During this process, there may be a report along the lines of

\begin{Verbatim}
srun: got SIGCONT
\end{Verbatim}

This is normal and can be ignored.

In the interface, messages from a particular processor are indicated by prepending the MPI rank of the processor in square brackets. For example,

\begin{Verbatim}
[0] Done.
\end{Verbatim}

is a message from processor 0. Messages from multiple processors are indicated with ranges, e.g. \texttt{[0-4,6,8-9]} indicates processors 0 through 4, processor 6, and processors 8 and 9.

Commands can be sent to a particular processor or subset of processors by prefixing the command with the \texttt{proc} command. For example,

\begin{Verbatim}
proc 0 print "Hello, world!"
\end{Verbatim}

will print ``Hello, world!'' on processor 0. More details on the \texttt{proc} command can be found in the reference section.

Output is ``deduplicated'', so similar output from multiple processors is printed only once. To expand, use the \texttt{expand} command (further documented in the reference section).

There are some important things to note regarding the interface. First, unlike in GDB, pressing control-c does not interrupt the application being debugged. Secondly, the interface does not have readline support and is therefore not ``fancy''. If you find that backspace does not work, using control-h should work instead. Lastly, not all commands function exactly the same way as in regular GDB. There is on-going work to normalize the interface towards the standard GDB interface.

To quit PGDB, type \texttt{quit}.

\newpage

\section{Examples}

This section contains several examples to illustrate how to make use of the debugger and answer common use questions. To some extent, the build upon one another. For further details on commands, consult the command reference. Most of these examples make use of Kull. Some of the output has been reformatted slightly to display better in this manual, or has been truncated for length, but is otherwise verbatim.

\subsection{Starting}

These examples illustrate methods of starting the debugger.

\subsubsection{Attach}

Suppose we have started a job that we wish to debug.

\begin{Verbatim}
srun -N 2 -n 4 -ppdebug kull test.py
\end{Verbatim}

First, we need to obtain the PID of the srun process that we used to launch the job. An easy way to do this is

\begin{Verbatim}
# ps x | grep srun
 96764 pts/134  Sl+    0:00 srun -N 2 -n 4 -ppdebug kull test.py
 97503 pts/16   S+     0:00 grep srun
\end{Verbatim}

The PID we need is the first one, $96764$. There may be other \texttt{srun} processes that appear in the list; make sure you choose the correct one.

Now, we start PGDB and attach to the process. This may take some time, as the debuggers need to load all the debug symbols.

\begin{Verbatim}
# pgdb -p 96746
GDB deployed to 2 hosts and 4 processors.
[0-3] [Thread debugging using libthread_db enabled]
[0-3] [New Thread 0x2aab45b9a700 (LWP 118885)]
[0-3] Stopped.
[0-3] 0x00002aab3f5a29cd in read() at ../sysdeps/unix/syscall-template.S:82
[0-3] Could not find ../sysdeps/unix/syscall-template.S
[0-3] Done.
Some results from [0-3] omitted; use expand to view.
\end{Verbatim}

We now have the application stopped where it was when we launched PGDB.

\subsubsection{Launch}

Alternately, suppose instead that we wish to launch an application directly under PGDB control. We can do this as follows.

\begin{Verbatim}
pgdb -a -N 2 -n 4 -ppdebug kull test.py
srun: got SIGCONT
GDB deployed to 2 hosts and 4 processors.
[0-3] Stopped.
[0-3] 0x00002aaaaaac3046 in memset() at ../sysdeps/x86_64/memset.S:555
[0-3] Could not find ../sysdeps/x86_64/memset.S
[0-3] Done.
Some results from [0-3] omitted; use expand to view.
\end{Verbatim}

We are now stopped in the initialization. However, this may be in some code prior to the \texttt{main} function that is typically thought of as the entry point. We can step forward to this as follows. We first set a breakpoint at \texttt{main} on all processors.

\begin{Verbatim}
break main
[0-3] Breakpoint 1 at <MULTIPLE>
\end{Verbatim}

In this case, there were multiple addresses for \texttt{main}, and hence why it was reported that the breakpoint is at multiple locations. Now, we continue until we hit the \texttt{main} function. This may take some time, because at this point, the debugger loads most of the debug symbols for shared libraries.

\begin{Verbatim}
continue
[0-3] Thread ID all running.
[0-3] Done.
[0-3] [Thread debugging using libthread_db enabled]
[0-3] Breakpoint 1:
[0-3] 0x000000000059dba1 in main(argc = 2, argv = 0x7fffffffe4a8) at
    kull/source/main.cc:98
[0-3] for (int ix= 1; ix<argc; ++ix) {
Some results from [0-3] omitted; use expand to view.
\end{Verbatim}

The program is now stopped in its \texttt{main} function, and debugging may proceed as normal.

\subsection{Breakpoints}

As illustrated above, breakpoints can be set with the \texttt{break} command. We cover a slightly more in-depth example of breakpoint usage here. First, we set the breakpoint and continue until we hit it.

\begin{Verbatim}
proc 0 break RadDiffusionEvolver.cc:125  
[0] Breakpoint 2 at <MULTIPLE>
continue
[0-3] Thread ID 1 running.
[0-3] Done.
Some results from [0-3] omitted; use expand to view.
[0] Breakpoint 2:
[0] 0x00002aaaf622c095 in RadDiffusion::RadDiffusionEvolver
    <Geometry::MeshBase<Geometry::XYZ>, Geometry::SideBase
    <Geometry::MeshBase<Geometry::XYZ> >, Geometry::CornerBase
    <Geometry::MeshBase<Geometry::XYZ> > >::addPartRosselandOpacity
    (this = 0x5e749a0, part = ..., pTe = ...) at
    kull/source/transport/RadDiffusion/RadDiffusionEvolver.cc:125
[0] Diffusion::Timer timer (this->timeRosselandOpacity);
\end{Verbatim}

To get information on the breakpoints that are presently set, we can do as follows:

\begin{Verbatim}
proc 0 info break
[0] Num Type       Disp Enb Address    What
[0] 1   breakpoint keep y   <MULTIPLE> in main
[0] breakpoint hit 1 time(s)
[0] 2   breakpoint keep y   <MULTIPLE> in RadDiffusionEvolver.cc:125
[0] breakpoint hit 1 time(s)
\end{Verbatim}

This tells us that we have two breakpoints, on in \texttt{main} and the other in \texttt{RadDiffusionEvolver.cc:125}. If we wish to delete a breakpoint, we use the \texttt{delete} command

\begin{Verbatim}
proc 0 delete 1
[0] Done.
\end{Verbatim}

\subsection{Examining Data}

There are many ways to examine data using PGDB. Here we provide three examples of the most common ways: backtraces, the \texttt{print} command, and the \texttt{varprint} command.

\subsubsection{Backtraces}

Backtraces are the most basic way to get information on where and how a bug happened. They provide the sequence of function calls that led to the current location in the code. To get a backtrace, do the following:

\begin{Verbatim}
proc 0 backtrace
[0] #00x00002aaaf622c095 in RadDiffusion::RadDiffusionEvolver
    <Geometry::MeshBase<Geometry::XYZ>, Geometry::SideBase
    <Geometry::MeshBase<Geometry::XYZ> >, Geometry::CornerBase
    <Geometry::MeshBase<Geometry::XYZ> > >::addPartRosselandOpacity
    at kull/source/transport/RadDiffusion/RadDiffusionEvolver.cc:125
[0] #10x00002aaaf622b382 in RadDiffusion::RadDiffusionEvolver
    <Geometry::MeshBase<Geometry::XYZ>, Geometry::SideBase
    <Geometry::MeshBase<Geometry::XYZ> >, Geometry::CornerBase
    <Geometry::MeshBase<Geometry::XYZ> > >::calcOpacitiesImp
    at kull/source/transport/RadDiffusion/RadDiffusionEvolver.cc:272
[0] #20x00002aaaf62229ef in RadDiffusion::RadDiffusionEvolverBase
    <Geometry::MeshBase<Geometry::XYZ> >::calcOpacities at
    kull/source/transport/RadDiffusion/RadDiffusionEvolverBase.cc:856
[0] #30x00002aaaf621d783 in RadDiffusion::RadDiffusionEvolverBase
    <Geometry::MeshBase<Geometry::XYZ> >::calcFinalUnknowns at
    kull/source/transport/RadDiffusion/RadDiffusionEvolverBase.cc:146
[0] #40x00002aaaf620914e in RadDiffusion::RadDiffusionModel
    <Geometry::MeshBase<Geometry::XYZ> >::advance at
    /kull/source/transport/RadDiffusion/RadDiffusionModel.cc:62
[0] #50x00002aab1f0e8648 in
    _wrap_RealPolyhedralRadDiffusionModel_advance__SWIG_0
    at polyhedralraddiffusionswig.cc:13061
[0] #60x00002aab1f0e8af2 in _wrap_RealPolyhedralRadDiffusionModel_advance
    at polyhedralraddiffusionswig.cc:13154
[0] #70x00000000016ed3e3 in PyObject_Call at Objects/abstract.c:2492
\end{Verbatim}

This backtrace output is truncated for length, but illustrates the point. Each line corresponds to a function call, with the top being the most recent one.

\subsubsection{Print}

Making use of the \texttt{print} command to examine and modify the values stored in variables is easy. The \texttt{print} command will support evaluating arbitrary expressions involving variables in the present scope.

To print out the value of a variable,

\begin{Verbatim}
proc 0 print argc
[0] 2
\end{Verbatim}

Now, suppose we want to perform some simple operations:

\begin{Verbatim}
proc 0 print argc * 2
[0] 4
proc 0 print argv[argc - 1]
[0] 0x7fffffffde28 \"test.py\"
\end{Verbatim}

We can also use \texttt{print} to assign values to variables.

\begin{Verbatim}
proc 0 print argc = 4
[0] 4
\end{Verbatim}

If we wish to print out a class, the syntax is the same:

\begin{Verbatim}
proc 0 print pTe
[0] @0x7fffffffaf50
\end{Verbatim}

However, in this case, \texttt{pTe} is a reference to a class, and hence only the address is printed. Fields of the class can be accessed as normal.

\begin{Verbatim}
proc 0 print pTe.mToken
[0] {px = 0x6b5e8f0 \"\\b_\\205?\\253*\", pn = {pi_ = 0x6b5e910}}
\end{Verbatim}

If you want to explore the structure of the class further, see the following section on \texttt{varprint}.

\subsubsection{Varprint}

The \texttt{varprint} command is a PGDB command to explore the structure of structured data such as classes. It will list the members of the class recursively, up to a depth limit, and will also avoid listing all the elements in large structures unless explicitly requested, to avoid an overwhelming amount of output. It works as follows:

\begin{Verbatim}
proc 0 varprint pTe.mToken
[0] boost::shared_ptr<char> mToken = {
   char * px = {
      char *px = 8 '\\b'
   }
   boost::detail::shared_count pn = {
      boost::detail::sp_counted_base * pi_ = 0x6b5e910
   }
}
\end{Verbatim}

Indentation is used to indicate the class hierarchy. If STL containers are present, they will be pretty-printed intelligently in order to show their contents. \texttt{varprint} is especially useful for exploring the relationship among larger structures.

To modify structures as printed by \texttt{varprint}, either the \texttt{print} command can be used as above, or alternately, the \texttt{varassign} command can be used.

\begin{Verbatim}
proc 0 varassign pTe.mToken.px = 0
[0] 0x0
\end{Verbatim}

\subsection{Execution Control}

Execution control provides for manipulation of the running of the program, in order to debug. We provide a quick overview of some of the basic commands for accomplishing this.

One of the most basic of such commands is \texttt{continue} which was demonstrated in prior examples. \texttt{continue} will cause the program to resume execution as normal starting from the current point at which it is stopped.

When the program is stopped, there are several other useful commands. The \texttt{step} command will step the program until it reaches a different source line. If there are function calls, the \texttt{step} command will descend into them.

\begin{Verbatim}
step
[0-3] Thread ID 1 running.
[0-3] Done.
[0-3] 0x00002aaacc66a6c4 in KullUtil::NullTimer::
    NullTimer(this = 0x7fffffffabff, counter = @0x6eea140)
    at kull/source/utilities/KullUtilTimer.hh:75
[0-3] NullTimer (double &counter) {}
Some results from [0-3] omitted; use expand to view.
\end{Verbatim}

Similar to the \texttt{step} command is the \texttt{next} command, which functions the same, except that it will treat function calls as one instruction and not descend into them.

Both the \texttt{step} and \texttt{next} commands take an optional integer argument which tells how many times to execute the command.

Finally, the \texttt{finish} command can be used to execute until the current function is exited.

\begin{Verbatim}
finish
[0-3] Thread ID all running.
[0-3] Done.
[0-3] Function finished, returned void.
[0-3] 0x00002aaaf622b382 in RadDiffusion::RadDiffusionEvolver
    <Geometry::MeshBase<Geometry::XYZ>, Geometry::SideBase
    <Geometry::MeshBase<Geometry::XYZ> >, Geometry::CornerBase
    <Geometry::MeshBase<Geometry::XYZ> > >::calcOpacitiesImp
    (this = 0x6ee9690, RosselandOnly = false)
    at kull/source/transport/RadDiffusion/RadDiffusionEvolver.cc:273
[0-3] if (!RosselandOnly) addPartPlanckianOpacity(*partItr, rPlanckianTe);
Some results from [0-3] omitted; use expand to view.
\end{Verbatim}

\newpage

\section{Command Reference}

\textbf{Note:} This section is in the process of being updated. It is mostly accurate, but incomplete.

This section provides an overview of all of the commands supported by PGDB. The vast majority of these are very similar to regular GDB commands and the commands in the GDB Machine Interface. For further details on GDB, refer to its documentation. This section does not attempt to detail every nuance of every command.

Note that some commands, in particular the uncommonly-used or obscure ones, may have had relatively little or no testing, and additionally may not have pretty-printer support. If you encounter such problems, report them on the issue tracker.

\subsection{Breakpoint Commands}

\lmargintt{ignore}
\texttt{ignore \emph{number} \emph{count}} \newline
Ignore breakpoint \ttemph{number} until it has been hit \ttemph{count} times.

\lmargintt{commands}
\texttt{commands \emph{number} [ \emph{command1} \ldots \emph{commandN} ]} \newline
Specify commands \ttemph{command1}, \ldots, \ttemph{commandN} to be executed when breakpoint \ttemph{number} is hit.

\lmargintt{condition}
\texttt{condition \emph{number} \emph{expr}} \newline
Breakpoint \ttemph{number} will stop the program only if \ttemph{expr} is true.

\lmargintt{delete}
\texttt{delete ( \emph{number} )+} \newline
Delete breakpoint(s) \ttemph{number}(s).

\lmargintt{disable}
\texttt{disable ( \emph{number} )+} \newline
Disable breakpoint(s) \ttemph{number}(s).

\lmargintt{enable}
\texttt{enable ( \emph{number} )+} \newline
Enable disabled breakpoint(s) \ttemph{number}(s).

\lmargintt{info break}
\texttt{info break [ \emph{number} ]} \newline
Get information on a single breakpoint \ttemph{number} or on all breakpoints.

\lmargintt{break}
\texttt{break [ -t ] [ -h ] [ -f ] [ -d ] [ -a ] [ -c \emph{condition} ] \newline [ -i \emph{ignore-count} ] [ -p \emph{thread-id} ] [ \emph{location} ]} \newline
Insert a breakpoint at \ttemph{location}, if specified. The optional parameters are:
\begin{description}
\item[\texttt{-t}] Temporary breakpoint.
\item[\texttt{-h}] Hardware breakpoint.
\item[\texttt{-f}] Create a pending breakpoint.
\item[\texttt{-d}] Create a disabled breakpoint.
\item[\texttt{-a}] Create a tracepoint.
\item[\texttt{-c}] Conditional breakpoint on \ttemph{condition}.
\item[\texttt{-i}] Set the ignore count to \ttemph{ignore-count}.
\item[\texttt{-p}] Restrict to \ttemph{thread-id}.
\end{description}

\lmargintt{list break}
\texttt{list break} \newline
List breakpoints. Same as \texttt{info break} (with no parameters).

\lmargintt{passcount}
\texttt{passcount \emph{tracepoint-number} \emph{passcount}} \newline
Set the passcount for tracepoint \ttemph{tracepoint-number} to \ttemph{passcount}.

\lmargintt{watch}
\texttt{watch [ -a | -r ] \emph{expr}} \newline
Create a watchpoint for \ttemph{expr}. \texttt{-a} creates an access watchpoint; \texttt{-r} creates a read watchpoint.

\lmargintt{awatch}
\texttt{awatch \emph{expr}} \newline
Create an access watchpoint for \ttemph{expr}.

\lmargintt{rwatch}
\texttt{rwatch \emph{expr}} \newline
Create a read watchpoint for \ttemph{expr}.

\subsection{Program Context Commands}

\lmargintt{set args}
\texttt{set args \emph{args}} \newline
Set inferior arguments to \texttt{args}.

\lmargintt{cd}
\texttt{cd \emph{dir}} \newline
Set the working directory to \ttemph{dir}.

\lmargintt{dir}
\texttt{dir [ -r ] [ \emph{path} ]+} \newline
Add directories \ttemph{path} to the beginning of the source file search path.
\begin{description}
\item[\texttt{-r}] Reset the search path first.
\end{description}

\lmargintt{path}
\texttt{path [ -r ] [ \emph{path} ]+} \newline
Add directories \ttemph{path} to the beginning of the object file search path.
\begin{description}
\item[\texttt{-r}] Reset the search path first.
\end{description}

\lmargintt{pwd}
\texttt{pwd}
Show the current working directory.

\subsection{Thread Commands}

\lmargintt{info thread}
\texttt{info thread [ \emph{thread-id} ]} \newline
Get information about \ttemph{thread-id} if present, or all threads.

\lmargintt{thread}
\texttt{thread \emph{thread-id}} \newline
\deprecated{Pass the \texttt{--thread} option to commands instead.}
Make \ttemph{thread-id} the current thread.

\subsection{Program Execution Commands}

\lmargintt{continue}
\texttt{continue [ --reverse ] [ --all | --thread-group \emph{N} ]} \newline
Resume execution.
\begin{description}
\item[\texttt{--reverse}] Resume execution in reverse.
\item[\texttt{--all}] Resume all threads. Ignored in all-stop mode.
\item[\texttt{--thread-group}] Resume all threads in thread group \ttemph{N}.
\end{description}

\lmargintt{finish}
\texttt{finish [ --reverse ]} \newline
Resume execution until the current function is exited.
\begin{description}
\item[\texttt{--reverse}] Resume execution in reverse until the point where the current function was called.
\end{description}

\lmargintt{interrupt}
\texttt{interrupt [ --all | --thread-group \emph{N}} \newline
Interrupt the execution of the target.
\begin{description}
\item[\texttt{--all}] Interrupt all threads.
\item[\texttt{--thread-group}] Interrupt all threads in thread group \ttemph{N}.
\end{description}

\lmargintt{jump}
\texttt{jump \emph{location}} \newline
Resume execution at \ttemph{location}.

\lmargintt{next}
\texttt{next [ --reverse ]} \newline
Resume execution, stopping when the next source line is reached.
\begin{description}
\item[\texttt{--reverse}] Resume execution in reverse, stopping when the prior source line is reached.
\end{description}

\lmargintt{nexti}
\texttt{nexti [ --reverse ]} \newline
Execute one machine instruction, treating a call as one instruction.
\begin{description}
\item[\texttt{--reverse}] Resume execution in reverse, stopping at the prior instruction.
\end{description}

\lmargintt{return}
\texttt{return}
Make the current function immediately return.

\lmargintt{run}
\texttt{run [ --all | --thread-group \emph{N} ]} \newline
Start execution from the beginning.
\begin{description}
\item[\texttt{--all}] Start all inferiors.
\item[\texttt{--thread-group}] Start thread group \ttemph{N}, which must be of type \texttt{process}.
\end{description}

\lmargintt{step}
\texttt{step [ --reverse ]} \newline
Resume execution, stopping when the next source line is reached. If there is a function call, stop at the first instruction in it.
\begin{description}
\item[\texttt{--reverse}] Resume execution in reverse, stopping at the beginning of the prior source line.
\end{description}

\lmargintt{stepi}
\texttt{stepi [ --reverse ]} \newline
Execute one machine instruction.
\begin{description}
\item[\texttt{--reverse}] Execute the prior machine instruction.
\end{description}

\lmargintt{until}
\texttt{until [ \emph{location} ]} \newline
Execute until \ttemph{location} is reached. If it is not specified, execute until a source line greater than the current one is reached.

\subsection{Stack Manipulation Commands}

\lmargintt{info frame}
\texttt{info frame} \newline
Get information on the selected frame.

\lmargintt{info stack depth}
\texttt{info stack depth [ \emph{max-depth} ]} \newline
Return the depth of the stack, not counting more than \ttemph{max-depth} frames if specified.

\lmargintt{list stack arguments}
\texttt{list stack arguments \emph{print-values} [ \emph{low-frame} \emph{high-frame} ]} \newline
Display arguments for frames between \ttemph{low-frame} and \ttemph{high-frame}, or the whole stack if they are not specified. \ttemph{print-values} should be 0, 1, or 2 for names, names and values, or simple values, respectively.

\lmargintt{backtrace}
\texttt{backtrace [ \emph{low-frame} \emph{high-frame} ]} \newline
List frames on the stack, within \emph{low-frame} and \emph{high-frame} if specified.

\lmargintt{list locals}
\texttt{list locals \emph{print-values}} \newline
List local variables in the selected frame. \ttemph{print-values} should be 0, 1, or 2 for names, names and values, or simple values, respectively.

\lmargintt{info locals}
\texttt{info locals \emph{print-values}} \newline
List local variables and function arguments in the selected frame. \ttemph{print-values} should be 0, 1, or 2 for names, names and values, or simple values, respectively.

\lmargintt{frame}
\texttt{frame \emph{frame}} \newline
\deprecated{Pass the --frame option to every function instead.}
Change the selected frame to \ttemph{frame}.

\subsection{Variable Objects Commands}

\lmargintt{enable pretty printing}
\texttt{enable pretty printing} \newline
Allow Python-based visualizers to affect the output of commands. This cannot be disabled once activated and is activated by default.

\lmargintt{var create}
\texttt{var create ( \emph{name} | - ) ( \emph{frame-addr} | * | @ ) \emph{expr}} \newline
Create a variable object to monitor \ttemph{expr}. It has name \ttemph{name}, which must be unique, or an automatically generated string if `\texttt{-}' is given. \ttemph{frame-addr} is the frame under which \ttemph{expr} is evaluated; a `\texttt{*}' indicates the current frame; a `\texttt{@}' indicates a floating variable object.

\lmargintt{var delete}
\texttt{var delete [ -c ] \emph{name}} \newline
Delete variable object \ttemph{name} and all of its children.
\begin{description}
\item[\texttt{-c}] Only delete the children of \ttemph{name}.
\end{description}

\lmargintt{var set format}
\texttt{var set format \emph{name} \emph{format-spec}} \newline
Set the output format \ttemph{format-spec} for variable object \ttemph{name}. \ttemph{format-spec} is specified as follows:

\texttt{format-spec ==> \newline( binary | decimal | hexadecimal | octal | natural )}

\lmargintt{var show format}
\texttt{var show format \emph{name}} \newline
Show the format for variable object \ttemph{name}.

\lmargintt{var info num children}
\texttt{var info num children \emph{name}} \newline
Return the number of children of variable object \ttemph{name}.

\lmargintt{var list children}
\texttt{var list children [ \emph{print-values} ] \emph{name} [ \emph{from} \emph{to} ]} \newline
Return a list of children of variable object \ttemph{name}, creating variable objects for them if needed. \ttemph{print-values} is 0, 1, or 2 for variable names, variable names and values, or variable names and simple values, respectively. If \ttemph{from} and \ttemph{to} are specified, reprot children in that range.

\lmargintt{var info type}
\texttt{var info type \emph{name}} \newline
Return the type of variable object \ttemph{name}.

\lmargintt{var info expression}
\texttt{var info expression \emph{name}} \newline
Return a user-readable string representing variable object \ttemph{name}.

\lmargintt{var info path expression}
\texttt{var info path expression \emph{name}} \newline
Return an expression for the full path to variable object \ttemph{name}.

\lmargintt{var show attributes}
\texttt{var show attributes \emph{name}} \newline
List the attributes of variable object \ttemph{name}.

\lmargintt{var evaluate expression}
\texttt{var evaluate expression [ -f \emph{format-spec} ] \emph{name}} \newline
Evaluates \ttemph{name} and returns its value.
\begin{description}
\item[\texttt{-f}] Return the value formatted according to \ttemph{format-spec}.
\end{description}

\lmargintt{var assign}
\texttt{var assign \emph{name} \emph{expr}} \newline
Assign the value of \ttemph{expr} to variable object \ttemph{name}.

\lmargintt{var update}
\texttt{var update [ \emph{print-values} ] ( \emph{name} | * )} \newline
Re-evaluate expressions corresponding to variable object \ttemph{name} and all its children, or all variable objects if `\texttt{*}' is given. \ttemph{print-values} is 0, 1, or 2 for variable names, variable names and values, or variable names and simple values, respectively.

\lmargintt{var set frozen}
\texttt{var set frozen \emph{name} \emph{flag}} \newline
Set the frozen flag on variable object \ttemph{name} to \ttemph{flag}. \ttemph{flag} should be `\texttt{0}' or `\texttt{1}' for unfrozen and frozen, respectively.

\lmargintt{var set update range}
\texttt{var set update range \emph{name} \emph{from} \emph{to}} \newline
Set the range of children to be returned on future updates of variable object \ttemph{name} to \ttemph{from} to \ttemph{to}.

\lmargintt{var set visualizer}
\texttt{var set visualizer \emph{name} \emph{visualizer}} \newline
Set the visualizer for variable object \ttemph{name} to be \ttemph{visualizer}. \ttemph{visualizer} should be a Python expression or `\texttt{None}'.

\subsection{Data Manipulation Commands}

\lmargintt{disassemble}
\texttt{disassemble [ -s \emph{start-addr} -e \emph{end-addr} ] \newline | [ -f \emph{filename} -l \emph{linenum} [ -n \emph{lines} ] ] -- \emph{mode}} \newline
Disassemble a location. \ttemph{mode} specifies the output: 0 for disassembly, 1 for mixed source and disassembly, 2 for disassembly with raw opcodes, or 3 for mixed source and disassembly with raw opcodes.
\begin{description}
\item[\texttt{-s}] Specify the beginning address \ttemph{start-addr}.
\item[\texttt{-e}] Specify the ending address \ttemph{end-addr}.
\item[\texttt{-f}] Specify the file \ttemph{filename} to disassemble.
\item[\texttt{-l}] Specify the line number \ttemph{linenum} to disassemble around.
\item[\texttt{-n}] Specify the number of disassembly lines \ttemph{lines} to produce. Specify $-1$ to disassemble the entire function.
\end{description}

\lmargintt{print}
\texttt{print \emph{expr}} \newline
Evaluate \ttemph{expr}.

\lmargintt{list changed registers}
\texttt{list changed registers} \newline
List the registers that have changed.

\lmargintt{list register names}
\texttt{list register names [ ( \emph{regno} ) + ]} \newline
List the register names for the current target. With no arguments, lists all registers. If integer arguments are given, list the names of registers corresponding to the arguments.

\lmargintt{info reg}
\texttt{info reg \emph{fmt} [ ( \emph{regno} )* ]} \newline
Display register contents. \ttemph{regno} is a list of registers to list contents for; if it is absent, all registers will be listed. \ttemph{fmt} is the format used for the contents:
\begin{description}
\item[x] Hexadecimal
\item[o] Octal
\item[t] Binary
\item[d] Decimal
\item[r] Raw
\item[N] Natural
\end{description}

\lmargintt{x}
\texttt{x [ -o \emph{byte-offset} ] \emph{address} \emph{count}} \newline
Read \ttemph{count} bytes from memory starting from \ttemph{address}.
\begin{description}
\item[\texttt{-o}] Specify a byte offset \ttemph{byte-offset} relative to \ttemph{address} from which to start reading.
\end{description}

\lmargintt{write memory bytes}
\texttt{write memory bytes \emph{address} \emph{contents}} \newline
Write hex-encoded bytes \ttemph{contents} to \ttemph{address}.

\subsection{Tracepoint Commands}

\lmargintt{tfind}
\texttt{tfind \emph{mode} [ \emph{parameters} \ldots ]} \newline
Find a trace frame based on \ttemph{mode} and the corresponding \ttemph{parameters}. Possible modes are:
\begin{description}
\item[\texttt{none}] No parameters. Stops examining trace frames.
\item[\texttt{frame-number}] Requires an integer. Selects tracepoint frames with that index.
\item[\texttt{tracepoint-number}] Requires an integer. Finds the next trace frame corresponding to the tracepoint with that number.
\item[\texttt{pc}] Requires an address. Finds the next trace frame corresponding to any tracepoint at that address.
\item[\texttt{pc-inside-range}] Requires two addresses. Finds the next trace frame corresponding to a tracepoint at an address in that range, inclusive.
\item[\texttt{pc-outside-range}] Requires two addresses. Finds the next trace frame corresponding to a tracepoint at an address outside that range, inclusive.
\item[\texttt{line}] Requires a line specification. Finds the next trace frame that corresponds to a tracepoint at that location.
\end{description}

\lmargintt{tvariable}
\texttt{tvariable \emph{name} [ \emph{value} ]} \newline
Create trace variable \ttemph{name} if it does not exist, with initial value \ttemph{value} if specified. \ttemph{name} should start with `\texttt{\$}'.

\lmargintt{tvariables}
\texttt{tvariables} \newline
Return a list of all trace variables.

\lmargintt{tsave}
\texttt{tsave [ -r ] \emph{filename}} \newline
Save trace data to \ttemph{filename}.
\begin{description}
\item[\texttt{-r}] If specified, ask the target to perform the save; if not, download data from target and save in a local file.
\end{description}

\lmargintt{tstart}
\texttt{tstart} \newline
Start tracing.

\lmargintt{tstatus}
\texttt{tstatus} \newline
Get the status of a trace.

\lmargintt{tstop}
\texttt{tstop} \newline
Stop a trace.

\subsection{Symbol Query Commands}

\lmargintt{list symbol lines}
\texttt{list symbol lines \emph{filename}} \newline
Print the list of lines that contain code and their associated program addresses for the given source file \ttemph{filename}.

\subsection{File Commands}

\lmargintt{file}
\texttt{file \emph{file}} \newline
Specify the executable \ttemph{file} to be debugged.

\lmargintt{exec file}
\texttt{exec file \emph{file}} \newline
Specify the executable \ttemph{file} to be debugged. Do not read the symbol table from \ttemph{file}.

\lmargintt{info source}
\texttt{info source} \newline
List the line number, current source file, and absolute path to the current source file for the current executable.

\lmargintt{info sources}
\texttt{info sources} \newline
List the source files for the current source file.

\lmargintt{symbol file}
\texttt{symbol file \emph{file}} \newline
Read symbol table information from \ttemph{file}.

\subsection{Target Manipulation Commands}

\lmargintt{attach}
\texttt{attach \emph{pid} | \emph{gid} | \emph{file}} \newline
Attach to a process \ttemph{pid}, thread group \ttemph{gid}, or file \ttemph{file}.

\lmargintt{detach}
\texttt{detach [ \emph{pid} | \emph{gid} ]} \newline
Detach from process \ttemph{pid}, thread group \ttemph{gid}, or the remote target if neither are specified.

\lmargintt{disconnect}
\texttt{disconnect} \newline
Disconnect from the remote target.

\lmargintt{load}
\texttt{load} \newline
Load the executable onto the remote target.

\lmargintt{target}
\texttt{target \emph{type} \emph{parameters} \ldots} \newline
Connect to the remote target. \ttemph{type} is the type of target (`\texttt{remote}', etc.) and \ttemph{parameters} are the associated parameters.

\subsection{File Transfer Commands}

\lmargintt{remote put}
\texttt{remote put \emph{hostfile} \emph{targetfile}} \newline
Copy local file \ttemph{hostfile} to \ttemph{targetfile} on the target system.

\lmargintt{remote get}
\texttt{remote get \emph{targetfile} \emph{hostfile}} \newline
Copy remote file \ttemph{targetfile} to \ttemph{hostfile} on the host system.

\lmargintt{remote delete}
\texttt{remote delete \emph{targetfile}} \newline
Delete \ttemph{targetfile} from the target system.

\subsection{Miscellaneous Commands}

\lmargintt{help}
\texttt{help \emph{topic}} \newline
Query the built-in GDB help system on topic \ttemph{topic}.

\lmargintt{quit}
\texttt{quit} \newline
Exit immediately.

\lmargintt{set}
\texttt{set \emph{expr}} \newline
Set a GDB internal variable based on \ttemph{expr}.

\lmargintt{show}
\texttt{show \emph{variable}} \newline
Display the current value of GDB variable \ttemph{variable}.

\lmargintt{show version}
\texttt{show version} \newline
Display GDB version information.

\lmargintt{list features}
\texttt{list features} \newline
Return a list of features the current MI protocol implements.

\lmargintt{list target features}
\texttt{list target features} \newline
Return a list of features supported by the target.

\lmargintt{list thread groups}
\texttt{list thread groups [ --available ] [ --recurse 1 ] [ \emph{group} \ldots ]} \newline
Lists information about thread groups \ttemph{group}(s), or all top-level thread groups.
\begin{description}
\item[\texttt{--available}] List information about thread groups available on the target.
\item[\texttt{--recurse}] List thread groups together with their children. A recursion depth of `\texttt{1}' must be specified.
\end{description}

\lmargintt{info os}
\texttt{info os [ \emph{type} ]} \newline
Return a list of available operating-system-specific information types. If \ttemph{type} is specified, return a available data of that type.

\lmargintt{add inferior}
\texttt{add inferior} \newline
Create a new inferior.

\lmargintt{interpreter exec}
\texttt{interpreter exec \emph{interpreter} \emph{expr}} \newline
Evaluate \ttemph{expr} under interpreter \ttemph{interpreter}.

\lmargintt{set inferior tty}
\texttt{set inferior tty \emph{tty}} \newline
Set the TTY of the current inferior to \ttemph{tty}.

\lmargintt{show inferior tty}
\texttt{show inferior tty} \newline
Show the TTY for the current inferior.

\lmargintt{enable timings}
\texttt{enable timings [ yes | no ]} \newline
Toggle printing of timing information for machine interface commands. No argument is equivalent to `\texttt{yes}'.

\subsection{PGDB-only Commands}

\lmargintt{varprint}
\texttt{varprint \emph{name}} \newline
Run the variable printer on variable \ttemph{name}.

\lmargintt{varassign}
\texttt{varassign \emph{name} = \emph{expr}} \newline
Assign \ttemph{expr} to variable \ttemph{name} that was created through \texttt{varprint}.

\lmargintt{proc}
\texttt{proc \emph{proc-spec} \emph{command}} \newline
Have only the remote nodes specified in \ttemph{proc-spec} execute command \ttemph{command}. \ttemph{proc-spec} may contain no spaces and is defined as follows:
\begin{itemize}
\item Specifications may be separated with commas.
\item A single integer specifies a single processor.
\item Two integers separated by a `\texttt{-}' specifies a range of processors, inclusive.
\end{itemize}

\lmargintt{block}
\texttt{block \emph{proc-spec}} \newline
Block output from the processors specified in \ttemph{proc-spec}. \ttemph{proc-spec} is defined above.

\lmargintt{unblock}
\texttt{unblock \emph{proc-spec}} \newline
Unblock output from the processors specified in \ttemph{proc-spec}. \ttemph{proc-spec} is defined above.

\lmargintt{filter}
\lmargintt{unfilter}
\deprecated{The \texttt{filter} and \texttt{unfilter} commands are undocumented and should not be used.}

\newpage

\section{Configuration Reference}

\textbf{Note:} This section is in the process of being updated. It is mostly accuracte, but incomplete.

This section provides an overview of all of the configuration variables within PGDB. The configuration files are stored as regular Python scripts inside the \texttt{conf/} directory.

\subsection{lmonconf}

These are configuration variables related to LaunchMON.

\lmargintt{use\_lmon\_10}
This determines which version of LaunchMON is used. Set to \texttt{True} to use LaunchMON 1.0, and \texttt{False} to use LaunchMON 0.7.2.

\lmargintt{lmon\_fe\_lib}
The path to the LaunchMON front-end library, which should be called \texttt{libmonfeapi.so}.

\lmargintt{lmon\_be\_lib}
The path to the LaunchMON back-end library, which should be called \texttt{libmonbeapi.so}.

\lmargintt{lmon\_version}
The version of the LaunchMON API, as defined in the \texttt{LMON\_VERSION} macro variable in the LaunchMON header files. This must match exactly the value in the installed version of LaunchMON.

\lmargintt{lmon\_environ}
A Python dictionary to use for setting LaunchMON-related environment variables. These settings are propogated to the back-end daemons' environments. Some environment variables that should probably be set are: \texttt{LMON\_REMOTE\_LOGIN}, \texttt{LMON\_PREFIX}, and \texttt{LMON\_LAUNCHMON\_ENGINE\_PATH}.

\subsection{gdbconf}

These are configuration variables related to PGDB in general, including GDB and MRNet.

\lmargintt{backend\_bin}
This is the binary to be executed by LaunchMON to run the back-end daemons. It should be your Python binary.

\lmargintt{backend\_args}
This is a list of arguments to be passed to the back-end daemons. The first argument should always be the path to \texttt{gdbbe.py}.

\lmargintt{environ}
This is a Python dictionary to use for setting environment variables that are needed for the functioning of PGDB. These settings are propogated to the back-end daemons' environments. Some environment variables that should probably be set are: \texttt{XPLAT\_RSH}, \texttt{MRNET\_COMM\_PATH}, and possibly \texttt{LD\_LIBRARY\_PATH}.

\lmargintt{topology\_path}
This is the path to the directory to create temporary topology files in, for the purpose of setting up MRNet.

\lmargintt{gdb\_init\_path}
This is the path to the \texttt{gdbinit} file that was included in the install of PGDB. It is provided as an initialization file to the GDB debuggers.

\lmargintt{pretty\_print}
This controls what kind of printing the PGDB interface does. The possible values are: \texttt{"yes"} for pretty-printing, \texttt{"no"} for raw printing, or \texttt{"both"} to do both.

\lmargintt{print\_dump\_file}
If this value is not \texttt{False} or \texttt{None}, this is the path to a file to dump raw printing output to, for debugging purposes. If this is set to a path, raw printing is dumped regardless of the setting of \texttt{pretty\_print}.

\lmargintt{varprint\_max\_depth}
This controls the default maximum depth to descend when \texttt{varprint} is printing a structure.

\lmargintt{varprint\_max\_children}
This controls the default maximum number of children that \texttt{varprint} will examine when printing a structure (unless the structure is explicitly printed, in which case this setting is ignored).

\lmargintt{mrnet\_branch\_factor}
This controls the branching factor to use when constructing the MRNet tree topology.

\lmargintt{topology\_transmit\_size}
This controls the size of each broadcast of topology data, from both the front-end to the back-end master, and from the back-end master to the other back-ends. This is used to split the data into smaller messages if it is too large.

\lmargintt{multi\_len}
This controls the maximum size of each message sent over MRNet. If a message is larger than this, it is split into multiple messages smaller than this value.

% These are to get the TOC to display the right page number and link to the right page.
\cleardoublepage
\phantomsection
\addcontentsline{toc}{section}{Index}
\printindex

\end{document}
